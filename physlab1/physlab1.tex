\documentclass[a4paper]{article}

%image insertion
\usepackage{graphicx} %image settings
\DeclareGraphicsExtensions{.pdf,.png,.jpg}

%math
\usepackage{amsmath} %math
%\usepackage{cmbright} %math font

%font
\usepackage{kotex}
\usepackage{fontspec}
\ifx가가
\setmainhangulfont[Ligatures=TeX,
BoldFont={KoPubBatang Medium}]{KoPubBatang Light}
\setsanshangulfont[Ligatures=TeX,
BoldFont={KoPubDotum Medium}]{KoPubDotum Light}
\setmainhanjafont[Ligatures=TeX,
BoldFont={KoPubBatang Medium}]{KoPubBatang Light}
\setsanshanjafont[Ligatures=TeX,
BoldFont={KoPubDotum Medium}]{KoPubDotum Light}
\xetexkofontregime[puncts=prevfont, colons=prevfont, cjksymbols=hangul]{latin}
\fi

%줄간격
\usepackage{setspace}
\usepackage{indentfirst}
\setstretch{1.3}
\everydisplay{\setstretch{1.2}}

%subfigure
\usepackage{subfigure}

\pagestyle{plain}
\title{물리 실험보고서 1}
\author{이한빈, 의예과 2016-13347}

\begin{document}


\numberwithin{equation}{section}
\maketitle

\section{Introduction}


	전기장은 전기력이 공간에 미치는 영역을 말하는 것으로 방향을 가지는 벡터이다. 
	즉, 한 점에서 전기장이 $\vec{E}(V/m)$로 주어졌을 때 그 점에 위치한 $+1C$의 전하가 크기는 $|\vec{E}|(N/C)$이고 방향은 $\vec{E}$인 전기력을 받는다는 것을 의미한다.
 	한편, 전하를 전기장에서의 한 점으로부터 다른 점까지 등속으로 옮기는 데 필요한 에너지는 그 경로에 무관한데 이로부터 전기적 퍼텐셜 에너지를 정의할 수 있다.
		\begin{equation}
			\Delta{} U = U_{1}-U_{2} = -\int_{r_{1}}^{r_{2}} \vec{E} \cdot d\vec{r} 
		\end{equation}



	위치의 미소 변화 $d\vec{r}$에 대해 위치에너지의 변화가 없을 경우($dU=0$) 위의 식으로부터 $\vec{E}\bot{}d\vec{r}$임을 알 수 있다. 
	따라서 각 점에서의 전기장을 연결한 전기장선과 퍼텐셜이 같은 점을 이은 등전위선은 수직하게 교차한다. 
	즉, 등전위선의 분포로부터 전기력선의 분포도 알 수 있다.

		\begin{figure}[htbp]
			\begin{center}
			    \includegraphics[width=\textwidth]{img/electircfield.png}
    			\caption{수직하게 교차하는 등전위선과 전기장선} \label{fig:ex1}
			\end{center}
		\end{figure}

	자유전자가 풍부한 도체에서는 전기장에서 평형을 이뤘을 때 모든 전하가 표면에 위치하고 내부 전기장은 0이 된다. 
	따라서 도체 주변의 전기력선은 도체 표면에 수직하게 들어간다.
\newpage

		\begin{figure}[htbp]
			\begin{center}
    			\includegraphics[width=0.5\textwidth]{img/dochae.png}
    			\caption{금속표면과 수직한 전기장선} \label{fig:ex2}
			\end{center}
		\end{figure}

	본 실험에서는 $+$와 $-$극에 사이의 전위를 측정하여 등전위선의 존재를 확인했으며 전극의 모양에 따라 등전위선의 모양이 어떻게 달라지는 지 알아보았다. 
	마지막으로 등전위선의 모양이 그렇게 나온 원인에 대해서 고찰하였다.

\section{Method}
	등전위선을 그리는 프로그램을 이용해 전극을 바꿔가며 등전위선을 얻는다.
	\subsection{점전극}
		준비물: 컴퓨터, 점 모양 나사전극, 전위발생판   

		전위발생판 전압은 최소, 프로그램의 전위간격은 $0.3V$으로 맞춘 후 프로그램 화면을 보면서 전극과 전극 사이의 모든 등전위선을 그린다. 
		전압을 중간, 최대로 맞춘 후 이 과정을 반복한다. 
		이 실험의 프로그램은 최초에 찍은 임의의 점과 같은 전위를 가진 점이 있으면 컴퓨터 화면에 같은 색깔의 점을 찍어주기 때문에 같은 색의 점들을 모으면 등전위선을 찾을 수 있다.

	\subsection{막대전극}
		준미물: 컴퓨터, 막대 모양 나사전극, 전위발생판

		전위 발생판 전압은 최대, 프로그램의 전위간격은 $0.5V$으로 맞춘 후 프로그램 화면을 보면서 전극과 전극 사이의 모든 등전위선을 그린다.
		등전위선을 찾는 원리는 위 실험과 동일하다.

\newpage

\section{Result}
	\subsection{점전극}

		\begin{figure}[h]
				\centering
					\subfigure[Minimum Voltage]{
					\includegraphics[width=0.4\textwidth]{img/electircfield0.png}
					\label{fig:subfig1}
					}
				\quad
					\subfigure[Midium Voltage]{
					\includegraphics[width=0.4\textwidth]{img/electircfield1.png}
					\label{fig:subfig2}
					}
				\\
					\subfigure[Maximum Voltage]{
					\includegraphics[width=0.4\textwidth]{img/electircfield2.png}
					\label{fig:subfig3}
					}
			\caption{Three experiments with point electrode with different voltages: anode on the left, cathode on the right}
		\end{figure}	

		전압이 높아짐에 따라 등전위선의 간격이 촘촘해지는 것을 관찰할 수 있다. 양 전극 가운데에는 등전위선이 쌍곡선 형태를 띠고 전극 가까이에서는 타원 형태를 띤다. 빨간색이 강할수록 전위가 높고, 파란색이 강할수록 전위가 낮은데 그림에서는 $+$극에서 $-$극에 가까워질수록 전위가 낮아지는 것을 관찰할 수 있다.   

	\subsection{막대전극}

		\begin{figure}[h]
			\begin{center}
				\includegraphics[width=0.4\textwidth]{img/electircfield-3.png}
				\caption{Experiment with rod electorde: anode on the left, cathode on the right. Rectangle on the each side is the rod. } \label{fig:fig4}
			\end{center}
		\end{figure}

		극 사이에서 등전위선이 평행함을 할 수 있다. 전극 밖으로 벗어나면 전극을 둘러싸는 형태로 등전위선이 휘어짐을 관찰할 수 있다. 
		앞선 실험과 마찬가지로 $+$극에서 $-$극에 가까워질수록 전위가 낮아진다.

\section{Conclusion}
	
	첫 번째 실험의 두 결과를 비교하면 입력 전압을 높였을 때 등전위선의 간격 수 있었다. 
	거리는 유지되지만 두 극사이의 전위차가 커짐에 따라 전위차를 일정하게 설정해놓은 등전위선의 밀도가 증가하는 것이다. 
	이는 등전위선의 밀도가 클수록 전기장이 강하다는 사실과 부합한다.

	첫 번째 실험에서 두 극의 중간 근처에서는 등전위선의 모양이 쌍곡선 형태로 나타났다. 이는 두 극 중간 근처에서 등전위선의 간격이 일정하게 나타났다는 점과 등전위선 및 쌍곡선의 정의로부터 알아낼 수 있다. 
	등전위선의 간격이 일정하다는 것은 등전위선 위의 점에서 각 극까지의 거리와 전위가 선형적인 관계를 가진다는 것이다. 
	따라서 두 점으로부터의 전위차가 일정한 등전위선은 두 점으로부터의 거리차가 일정한 점들의 집합이 되어 쌍곡선의 정의에 부합하게 됨을 알 수 있다.

	극에 가까워질수록 등전위선의 형태가 쌍곡선과 멀어진다. 
	이는 그 극 근처에서 등전위선의 간격이 급격하게 좁아지는 것으로부터 알 수 있다. 등전위선의 간격이 일정하게 유지되지 않고 전위와 극사이의 선형적인 관계가 없어짐에 따라서 첫 번재 문단에서 얘기한 규칙이 깨지기 때문이다. 
	더불어 극 근처에서는 한 쪽 극의 영향이 지배적이나 반대쪽 극의 영향 때문에 비대칭적인 원형을 띄게 된다.

	두 번째 실험에서 등전위선은 양 극판 사이에서는 극판과 평행하고 극판 밖에서는 극판을 둘러싸는 형태를 띄었다. 
	이는 극판의 형태를 따라서 등전위면이 형성된다는 것이므로 극판의 성분인 금속의 표면이 전기장 속에서 등전위면을 형성한다는 사실과 일치한다.

	마지막으로 각각의 실험에서 동전을 전기장 속에 놓고 주변에서 등전위선의 형태를 측정하고자 했으나 첫 번재로, 동전이 고정이 되지 않아서 정확한 측정을 하지 못했다. 
	두 번째로, 장치의 입력을 최대, 등전위선 간격을 최소로 설정했음에도 등전위선의 형태변화를 얻을 수 없었다. 이는 등전위선이 간격에 비해 동전의 크기가 작아서 등전위선이 동전에 의해 왜곡되더라도 왜곡된 등전위선은 컴퓨터 화면에 나타나지 않았기 때문이다. 
	이를 개선하기 위해서는 더 높은 출력을 가진 장치가 이용해 등전위선 사이의 간격을 동전의 크기보다 줄여야 될 것으로 예상된다.

\section{Reference}
	1. Halliday, D., Resnick, R., \& Walker, J. (2014). {\it{}Principles of Physics} (10th ed., Vol. 2). Hoboken, NJ: Wiley. 

\end{document} 

%실험에서 개선할 점 등 피피티에서 봤던 거 모두 적어서 처리합시다